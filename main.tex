%%%%%%%%%%%%%%%%%%%%%%%%%%%%%%%%%%%%%%%%%
% Developer CV
% LaTeX Template
% Version 1.0 (28/1/19)
%
% This template originates from:
% http://www.LaTeXTemplates.com
%
% Authors:
% Jan Vorisek (jan@vorisek.me)
% Based on a template by Jan Küster (info@jankuester.com)
% Modified for LaTeX Templates by Vel (vel@LaTeXTemplates.com)
%
% License:
% The MIT License (see included LICENSE file)
%
%%%%%%%%%%%%%%%%%%%%%%%%%%%%%%%%%%%%%%%%%

%----------------------------------------------------------------------------------------
%	PACKAGES AND OTHER DOCUMENT CONFIGURATIONS
%----------------------------------------------------------------------------------------


\documentclass[9pt]{developercv} % Default font size, values from 8-12pt are recommended


\usepackage[T1,T2A]{fontenc}
\usepackage[utf8]{inputenc}
\usepackage[english,russian]{babel}
\usepackage{libertine}

%----------------------------------------------------------------------------------------

\begin{document}

%----------------------------------------------------------------------------------------
%	TITLE AND CONTACT INFORMATION
%----------------------------------------------------------------------------------------

\begin{minipage}[t]{0.65\textwidth} % 45% of the page width for name
	\vspace{-\baselineskip} % Required for vertically aligning minipages

	% If your name is very short, use just one of the lines below
	% If your name is very long, reduce the font size or make the minipage wider and reduce the others proportionately
	% \colorbox{black}{} % First name
    {\HUGE\textbf{\MakeUppercase{Виногродский}}}

    {\HUGE\textbf{\MakeUppercase{Серафим}}}

	\vspace{6pt}

    {\HUGE\textbf{\MakeUppercase{Брониславович}}}

	\vspace{6pt}

	{\huge Студент бакалавриата} % Career or current job title
\end{minipage}
\begin{minipage}[t]{0.25\textwidth} % 27.5% of the page width for the first row of icons
	\vspace{-\baselineskip} % Required for vertically aligning minipages

	% The first parameter is the FontAwesome icon name, the second is the box size and the third is the text
	% Other icons can be found by referring to fontawesome.pdf (supplied with the template) and using the word after \fa in the command for the icon you want
    \icon{BirthdayCake}{12}{19.07.2003}\\
	\icon{MapMarker}{12}{Москва, Россия}\\
	\icon{Phone}{12}{+7 915 177--99--90}\\
	\icon{At}{12}{\href{mailto:fimmind@mail.ru}{fimmind@mail.ru}}\\
	\icon{Github}{12}{\href{https://github.com/fimmind}{github.com/fimmind}}\\
\end{minipage}

\vspace{0.5cm}

%----------------------------------------------------------------------------------------
%	INTRODUCTION, SKILLS AND TECHNOLOGIES
%----------------------------------------------------------------------------------------

\cvsect{Обо мне}

\begin{minipage}[t]{0.4\textwidth} % 40% of the page width for the introduction text
    Высоко ответственный и организованный студент 2-го курса бакалавриата по направлению <<Математика>>, чья главная задача --- это достижение по-настоящему глубокого понимания в интересующих меня разделах математики. Для приближения к этой цели я стараюсь постоянно совершенствоваться и экспериментировать с новыми и оригинальными подходами к учёбе и самообразованию. Так же интересуюсь программированием и теоретической информатикой.
\end{minipage}
\hfill % Whitespace between
\begin{minipage}[t]{0.5\textwidth} % 50% of the page for the skills bar chart
	\vspace{-\baselineskip} % Required for vertically aligning minipages
	\begin{barchart}{5.5}
        \baritem{Anki}{100}
        \baritem{LaTeX}{80}
        \baritem{Vim}{80}
        \baritem{Linux}{80}
        \baritem{Git}{50}
        \baritem{Python}{60}
        \baritem{Rust}{40}
        \baritem{Haskell}{30}
	\end{barchart}
\end{minipage}

%----------------------------------------------------------------------------------------
%	EDUCATION
%----------------------------------------------------------------------------------------

\cvsect{Образование}

\begin{entrylist}
	\entry
		{2021 -- (2025)}
		{Бакалавриат, направление <<Математика>>}
		{Российский Университет Дружбы Народов}
		{За больше чем полтора года обучения я не получил ни одной оценки, ниже пятёрки. По-началу поддерживать такую успеваемость было тяжело, однако в процессе я смог выработать для себя чёткую стратегию подхода к обучению, основанную в первую очередь на технике интервального повторения. Как результат, все без исключения профильные предметы сессии последнего (третьего) семестра были с заведомой лёгкостью закрыты мной на максимальный балл (100 из 100 по шкале ECTS).}
    \entry
        {2020 -- 2021}
        {11 класс}
        {Онлайн Школа <<Фоксфорд>>}
        {Последние школьные годы я по собственному желанию проходил сначала в заочном, а одиннадцатый класс и вовсе в онлайн формате. Это дало мне бесценный опыт самоорганизации, который в дальнейшем значительно упростил для меня процесс адаптации к вузовскому обучению.}
    \entry
        {2010 -- 2020}
        {1 -- 10 классы}
        {ГБОУ СОШ <<Школа №1248>>}
        {}
\end{entrylist}

%----------------------------------------------------------------------------------------
%	ADDITIONAL INFORMATION
%----------------------------------------------------------------------------------------

\begin{minipage}[t]{0.3\textwidth}
	\vspace{-\baselineskip} % Required for vertically aligning minipages

	\cvsect{Языки}

	\textbf{Русский} - родной\\
	\textbf{Английский} - продвинутый\\
	\textbf{Японский} - начальный
\end{minipage}
\hfill
\begin{minipage}[t]{0.6\textwidth}
	\vspace{-\baselineskip} % Required for vertically aligning minipages

	\cvsect{Хобби}

    Уже на протяжении 8 лет я регулярно три раза в неделю занимаюсь скалолазанием. В остальное свободное от вуза время я занимаюсь в основном самостоятельным изучением японского языка и абстрактной алгебры. Когда есть возможность, так же самостоятельно изучаю программирование, например, по книге Structure and Interpretation of Computer Languages.
\end{minipage}

%----------------------------------------------------------------------------------------

\end{document}
